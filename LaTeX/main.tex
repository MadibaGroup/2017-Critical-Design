\documentclass[manuscript]{acmart}


% Page Numbers
%\settopmatter{printfolios=true}

% Shrink copyright stuff
%\settopmatter{printacmref=false}
%\renewcommand\footnotetextcopyrightpermission[1]{}
%\usepackage{silence}


% = = = = = Packages = = = = = %

%% !TEX root = ../main.tex

%------------------------SOUPS----------------------%

% = = = SOUPS Specific
\usepackage{times}
\renewcommand{\topfraction}{0.99} % be more aggressive about text around floats
\renewcommand{\floatpagefraction}{0.99}
\pagestyle{plain} % page numbers

%------------------------Packages----------------------%

% = = = Graphics
\usepackage{graphicx}
\graphicspath{{figures/}}
\DeclareGraphicsExtensions{.jpg,.png}

% = = = Subfig (note not subfigure)
\usepackage[caption=false,font=footnotesize]{subfig}

% = = = Math Symbols
\usepackage{amsmath}
%\usepackage{amstext,amssymb,amsthm}
\usepackage{bbm}
\usepackage{stmaryrd}

% = = = Other
\usepackage{array}
\usepackage{color}
\usepackage[hyphens]{url}
\usepackage[pdftitle=Title,pdfauthor=Anonymous]{hyperref}

%------------------------END----------------------%  
  


% !TEX root = ../main.tex
%------------------------Custom Commands----------------------%

% = = = Latin Short-forms (ie, eg, etc, et al)
\usepackage{xspace}
\newcommand{\etal}{\textit{et al.}\xspace}
\newcommand{\etc}{\textit{etc.}\xspace}
\newcommand{\ie}{\textit{i.e.,}\xspace}
\newcommand{\eg}{\textit{e.g.,}\xspace}
\newcommand{\cf}{\textit{cf.}\xspace}
\newcommand{\supra}{\textit{Supra}\xspace}

% = = = Arrow -> (\lt)
\newcommand{\lt}{$\rightarrow$\xspace}

% = = = Keywords (kw)
\newcommand{\kw}[1]{\textsf{#1}}

% = = = Colored text (textblue)
\newcommand{\textblue}[1]{\textcolor{blue}{#1}}

% = = = Compact Lists (compactlist, compactlistn)
\newenvironment{compactlist}
  {\begin{itemize} 
  \setlength{\itemsep}{0pt} 
  \setlength{\parskip}{0pt}} 
  {\end{itemize}}
  
\newenvironment{compactlistn}
  {\begin{enumerate} 
  \setlength{\itemsep}{0pt} 
  \setlength{\parskip}{0pt}} 
  {\end{enumerate}}
  
\renewcommand{\labelitemi}{$\bullet$}
  

%------------------------Crypto----------------------%  

% = = = Zp, Gq and Zq
\newcommand{\Zp}{\mathbb{Z}^{*}_{p}}
\newcommand{\Zq}{\mathbb{Z}_{q}}
\newcommand{\Gq}{\mathbb{G}_{q}}

% = = = Encryption, etc.
\newcommand{\Enc}[1]{\mathsf{Enc}(#1)}
\newcommand{\EncB}[1]{\llbracket #1 \rrbracket}
\newcommand{\ReRand}[1]{\mathsf{ReRand}(#1)}
\newcommand{\Hash}[1]{\mathcal{H}(#1)}
\newcommand{\Sign}[1]{\mathsf{Sig}(#1)}
\newcommand{\Comm}[1]{\mathsf{Comm}(#1)}
\newcommand{\Open}[1]{\mathsf{Open}(#1)}

% = = = Tuples
\newcommand{\tuple}[1]{\left \langle #1 \right \rangle}


%-------------------Custom for Paper----------------------%

% = = = Name
\newcommand{\Name}{\textsf{System Name}\xspace}







% = = = = = Title = = = = = %

\begin{document}

\title{Security Noir: Critical Designs for Security \& Privacy}

\author{Didem Demirag}
\affiliation{\institution{UQAM} \city{Montreal} \state{QC} \country{Canada}}


\author{Jeremy Clark}
\email{j.clark@concordia.ca}
\orcid{0000-0002-3533-5965}
\affiliation{\institution{Concordia University} \city{Montreal} \state{QC} \country{Canada}}

\begin{abstract}

In this paper, we investigate an alternative method for communicating the somewhat elusive security risks and privacy invasions of technology, the web, social networks, and other aspects of modern life. Specifically, we draw from the \textit{critical design} movement popularized by Dunne and Raby in the 90s. In this project, we have conceptualized 10 new technological designs and described their use through short vignettes. These designs are imaginary but plausible technologies that illustrate a security or privacy risk through absurdity, satire, or humour. We also consider the ability to co-design with artificial intelligence tools, finding the tools cannot (yet?) replicate our design process and produce satisfactory designs, but these tools are useful today in a `co-pilot' role for generating high volumes of ideas for human designers to curate and refine. 

\end{abstract}

\maketitle


% = = = = = Main Body = = = = = %

% !TEX root = ../main.tex

\section{Introductory Remarks}

% !TEX root = ../main.tex

\begin{table*}
\centering
\caption{Examples of Critical Design\label{tab:examples}}
\begin{tabular}{|p{2.5cm}|p{11cm}|} 
\hline

Work & Vignette \\ \hline

\textit{Compass Table}, \newline A. Dunne \newline F. Raby &
``This table reminds you that electronic objects extend beyond their visible limits. The 25 compasses set into its surface twitch and spin when objects like mobile phones or laptop computers are placed on it. The twitching needles can be interpreted as being either sinister or charming, depending on the viewer's state of mind. When we designed the compass table, we wondered if a neat-freak might try to make all the needles line up, ignoring the architectural space of the room in favour of the Earth's magnetic field.'' Quoted from \cite{DuRa01}. \\ \hline

\textit{Life Counter}, \newline I. Matsumoto &
``With Life Counter (2001), you choose how many years you would like to or expect to live for and start the counter. Once activated, it counts down the selected time span at four different rates: the number of years, days, hours or seconds to go are shown on different faces. Depending on which face you choose to display, you may feel very relaxed as the years stretch out ahead or begin to panic as you see your life speed away before your eyes. The counter is designed to be visually unassuming and could easily fit into the slightly retro-futuristic style of the moment.'' Quoted from \cite{DuRa01}. \\ \hline

\textit{Open Informant}, \newline J Ardern \newline Y. Ushigome \newline A. Jain &
``Open Informant (2013) takes the form of a networked object including a phone app and e-ink badge. The app searches your communications for National Security Agency trigger words and then sends text fragments containing these words to the badge worn by the user for public display. Using the body as an instrument for protest, the badge becomes a means of rendering our own voice visible in an otherwise faceless technological panopticon.'' Quoted from \cite{Mal17}. \\ \hline

\end{tabular}
\end{table*}

Critical design is commonly attributed to Dunne and Raby~\cite{DuRa01,Dun05} however it can be seen as a refinement of earlier movements in arts and design, beginning with the Italian radical design movement and extending through critical practice in HCI~\cite{Mal17}. Consider the three examples provided in Table~\ref{tab:examples}. Each describes the design of an object or technology. In contrast to industrial design, these technologies are not oriented around problem-solving and the creation of solutions that are functional, useful, and pleasurable---they are not meant for mass production. Rather they offer a subtle criticism by illustrating something---electromagnetic radiation, the inevitability of death, the ubiquity and stealth of surveillance---that typically goes unquestioned. They are designed to invoke a dilemma for the user, raising questions, challenging values, and stimulating debate. Critical design considers these goals to be the function (or `para-function'~\cite{Dun05}) of the concept. %, enabling traditional design approaches to maximizing this functionality. 

Critical design has significant overlap with conceptual art, utilizing common methods, and has been disseminated through exhibitions and the art literature. However critical design works by showcasing things that are plausible, believable, and easy for the reader to situate, requiring design practice. Viewing critical design as art creates challenges. Consider the compass table (from Table~\ref{tab:examples}) in the context of an art gallery exhibit---it is meant to illustrate elctromagnatic radition in every day life, in one's home, in different positions and orientations within that home---all of this is lost when it sits statically in a gallery. Should critical design be displayed in galleries, a vignette or film or narrative will often accompany the object to describe its context of use.

%\paragraph{Relation to Science.} 
%
%Critical design is ideological. Science can be as well. Even cryptography, positioned as one of the most scientifically rigorous pursuits in the field of security and privacy, is ideological, as argued eloquently by Rogaway in a recent keynote~\cite{Rog15}. For example, he notes that work on bland-sounding subjects like differential privacy and identity-based encryption are premised on a certain values and world views---respectively, one of massive data collection and one of government-escrowed master decryption keys. His point is that work in these areas \textit{affirm} certain ideologies; it is \textit{affirmative} design and more cryptographers should consider what is it that their designs affirm?
%
%From a different angle, Herley and van Oorschot question which aspects of security, privacy, and cryptography are even a proper science~\cite{HvO17}. Critical design is certainly not an objective form of design where success can be measured. We purposely do not set out to `measure' the success of our designs as this deeply misses the point of critical design. While our designs are for scientists, engineers, and technologists, it is not to establish theorems or facts but to raise new questions and communicate risks and invasions in a new way.

\subsection{Contributions}

\paragraph{Security \& Privacy.} Security and privacy risks are often intangible and invisible, making concepts difficult to communicate to everyday users. We saw a parallel to electromagnetic radiation in this regard, and considered if critical design could similarly aid in illuminating risks. The primary contribution of this paper is a set of designs.

\paragraph{Vignettes.} Critical designs are commonly described through vignettes. Occasionally they are built and displayed. In the case of Dunne and Raby's Compass Table (and the seven other objects in their \textit{Placebo} project~\cite{DuRa01}), the table was built and given to individuals to use. This field study was followed by an exit survey on how the participants felt about the object. In this paper, we describe our designs through short vignettes that illustrate its design and use case. If they are a successful design, the reader should easily imagine its use in her life and question it critically, resulting in its `rhetorical use'~\cite{Mal17} rather than practical use. These designs are transgressive and intended to provoke the reader's emotions; invoking humour, satire, ridicule, poetry, playfulness, lewdness, appropriation, irreverence or deliberate overcompensation.

\paragraph{Methodology.} In order to generate designs, we initially employed a XXX methodology, using concepts and sticky notes (see Figure). The design process coincided with the emergence of machine learning tools, based on large language models, like ChatGPT. We decided to also experiment with AI-assistance. We conducted four experiments. First, we (a) used traditional design methodology (no AI) to generate 10 designs (Section~\ref{}). We then (b) fed the 10 designs to ChatGPT version GPT-4g and asked it to generate similar designs (Section~\ref{}). It effortlessly generated designs that initially seemed sound, but under greater scrutiny, were lacking. We then (c) experimented with giving ChatGPT a sketch of what a possible design could involve and then dialogued with it to develop and refine the design, using it is a collaborative tool (Section~\ref{}). We found this approach produced good results and was less effort than method (a). Finally we (d) educated ChatGPT on the entire design process (Section~\ref{}), trying to involve it in the earliest stages of design (as opposed to (c) where we involved it after we had an initial sketch of an idea). We found this approach was less effective. In our experience, ChatGPT is capable of assisting in designs but requires human guidance and intervention. AI's strength is in generating large volumes of ideas which can be curated and refined by humans.

\subsection{Related Work}

\paragraph{Security and Privacy in Critical Design}

Much has been written on critical design and Malpass provides an excellent survey~\cite{Mal17}. Since our focus is security and privacy, we only summarize the following concepts, situated within or close to critical design, since they deal with issues of privacy: 

\begin{compactlist}
\item \textbf{The Pillow.} Dunne and Gaver's design takes the form factor of a transparent pillow and plays radio signals it picks up from the technology of that time: \eg phones, pagers, and baby monitors~\cite{DuGa97}. It challenges users to consider how confidential these signals are.  
\item \textbf{Open Informant.} Described in Table~\ref{tab:examples}, this design exposes nation-state surveillance.
\item \textbf{United Mirco-Kingdoms.} Dunne and Raby reimagine the UK as four `supershires' where one is occupied by `digitarians:' extreme technocrats ruled by algorithms in a state of total surveillance.  
\item \textbf{Alternet.} While not critical design, Gold's Alternet has the similar intention of provoking thought through design~\cite{Gol14}. The internet is re-imagined with users retaining full control over their private data, touching on web tracking and our growing digital dossiers. 
\end{compactlist}

We are unaware of any attempt to build a systematic set of examples based on security or privacy, nor much attention to it from academic research papers. %Many critical designs are `one-off.' The idea of developing a set of concepts related by a common theme, rather than individual concepts, is however established. For example, the Placebo Project mentioned already offered 8 concepts  (including Compass Table in Table~\ref{tab:examples}) that were each related to EMF radiation. 

%  Some security and privacy research offers seemingly unintentional critical design. In this case, we subsume the designs in our examples while clearly noting when they originate from other researchers. We also include `found' examples of critical design from outside of academia. We believe that situating them alongside our original ideas allows the reader to consider how close our concepts are to reality. 

\paragraph{Security and Privacy in Alternative Pedagogy}

But there are works on finding alternative ways to explain security concepts: Design fiction~\cite{loureiro-koechlin_vision_2022}, role playing game~\cite{merrill_security_2020}, other fields like sensing technologies: science fiction and design fiction~\cite{wong_real-fictional_2017}


\paragraph{Critical Design in HCI}

Using AI in creative writing: \cite{chakrabarty_art_2024}, \cite{kim_authors_2024}, fashion~\cite{davis_fashioning_2024}

\paragraph{Methodologies for Design Creation} come up with the design manually like whiteboarding, sticky notes. 
Can also be AI-assisted: examining how to prompt generative AI~\cite{sanchez_examining_2023}, ~\cite{chang_prompt_2023}
\cite{huang_designnet_2023}: Grouping design aspects to help ideation and exploration of design solutions.

\cite{tholander_design_2023}: generative AI in creative design for ideation, early prototyping and sketching.

\cite{chiou_designing_2023}: co-ideation with image generation

\cite{lin_generative_2023}: speculative designs which are analyzed according to different concepts like  Environment, Data Privacy, Embodiment, and Play. They visualize designs through generative AI.


\paragraph{Methodologies for Design Evaluation} we only created the designs, 
user studies: ~\cite{sanchez_examining_2023}, ~\cite{chang_prompt_2023}
Custom-made AI for specific design goals:context aware human-AI co-creation\cite{fan_contextcam_2024}

\paragraph{Methodologies for Pedagodgy} different methods to teach concepts related to computer science in general or cybersecurity to a diverse audience. How CS students perceive AI and Cybersecurity~\cite{ojha_computing_2023}, visual cryptography for K-12 students~\cite{rayavaram_designing_2023}


% = = = = = = = =



\section{Methodology 1: Traditional Design}

\subsection{Methodology}

\subsection{Designs}

\design{Crystal Avatars}{Crystal avatars are created instantly but only gets better over time. At first, they are not recognizable and cannot be differentiated from other users, but as users browse the site, small details crystallize. Alice is happy to save time registering for the site and have a personal avatar, which automatically added a cat accessory after she spent a lunch hour reading about hairball treatments. However she grows uneasy about the "Woman, Life, Freedom" badge, a cause she believes in but only vocalizes discriminately.}



\design{Calm Watch}{Calm Watch is a smart watch that tracks anxiety levels based on several biomarkers. When the anxiety level of the wearer crosses a threshold, a haptic vibration pulse prompts the user to look at the watch. From there, they can select a variety of calming techniques, including breathing exercises and guided meditation. Bob is nervous to talk to Alice and becomes flustered when his watch starts vibrating mid-conversation, just loud enough for Alice to hear it and notice his fidgeting with the watch.}

\design{Chatty}{The latest software update for Alice's voice-activated home assistant Chatty adds machine learning to better infer Alice's commands, such as adding items to her shopping list and suggesting music she might like. It does not always know whether Alice is talking to it or not, so over time it picks up on pieces of Alice's life and has trouble unlearning them. Alice is startled but pleased when Chatty chimes in with the name of `that actor who was in that thing' she was telling her friend about. She is bemused to find a specialized vinyl cleaning solution on her shopping list after she said Bob's records smells fishy.}

\design{Dynamic Laughtrack}{Digital television content, such as sitcoms, encode laughtracks as a series of cues rather than an actual recording. Smart TVs listen and classify the viewer's level of laughter on a 10 point scale (with fine grain training over time). The TV adds laughter to the content in a sensitive and considerate manner, where the laughtrack is only marginally higher on the scale than the viewer's own laughter. This gently nudges the viewer to greater enjoyment of the program without bombarding her. It also mixes in actual past recordings of the viewer's own laughter to capitalize on emotional mirroring.}

\design{Eleventh Finger}{Biometric authentication based on fingerprints is generally user-friendly and fast. Eleventh finger is a 3D printed rubber finger, customized with the user's fingerprint. You can put it on a keychain and use it on cold winter days when you do not want to remove your gloves. Alice lets her friend Bob borrow it when he stays at her house. It can also serve as a backup if the worst happens.}

\design{GOTTCHA}{GOTTCHA is a human-detection system to protect online accounts being accessed by bots. It invokes the device camera to analyze if a live human is using the device. To protect against video replays or machine-generated video, it unexpectedly prompts the user with a randomly selected image and carefully measures their reaction to it. Micro-expressions are involuntary facial displays of emotion that are too fast to mimic artificially. GOTTCHA's image bank can provoke disgust, anger, fear, sadness, happiness, surprise or contempt. Over time, Alice stops visiting websites that use GOTTCHA because of its capricious tendency to mix in disturbing images.}

\design{No(i)sy elevator}{Elevator music has never been regarded as elevating your mood. This is why employees will be happier riding the noisy elevator. When it recognizes a user, it spins up a selection from their most-played songs on major music streaming platforms. Alice finds hearing her favourite Miles Davis song grounds her at the start of each work day, while Bob and Carol, coincidently riding the elevator together, discover their shared love for 90s britpop. David is a bit more concerned about the profanity ridden banger that the nosy elevator plays for him and his boss.}

\design{Receiver plant}{The receiver plant in the front yard of Alice records the information of the people that pass by. It is a greedy plant that needs to absorb enough bluetooth data to grow. Bob is singlehandedly responsible for its lustrous canopy from walking by Alice's house every day, mostly while checking his work emails on his phone. Perhaps the plant overstepped when it displayed, "you are 15 minutes late today, are you sure you can make it to work on time?"}

\design{ToS Fishing}{ToS Fishing is a digital game where players earn points by hunting down excessive legalese on websites, including terms of service (ToS), privacy policies, and cookie policies. To play, users simply copy and paste the URL of the legalese. If the ToS has been seen before, the users are awarded points based on its word count. If it has not been seen before, it is reviewed by a human for validity and word count—and in this case, the user gets a finder's bonus. To mitigate people from launching custom websites, the game only accepts sites from the Alexis top million. Users display their aggregate score on a leaderboard with a profile showing their top catches (word count only, not the website), and can earn various badges for playing consistently.}

\design{Wifi Projector}{Minimalist wifi routers have a single light indicator to show internet connectivity, such as a green light as opposed to red light. This understates the interesting and intriguing data flowing through the device as users browse the internet. By contrast, the wifi projector is a talkative router that projects visualizations onto the ceiling above where it is placed. Every website that pulls in scripts and cookies from other domains is displayed as a constellation of stars of various colours and sizes, which slowly fade away. While Alice's visit to dictionary.com is a beautiful galaxy, she is also perturbed by the number of ads and tracking companies surveilling her.}


\subsection{Discussion}


\section{Methodology 2: LLM Polish}

\subsection{Methodology}

\subsection{Designs}

\design{ElephantMind}{ElephantMind is an augmented reality (AR) social interaction app designed to enhance small talk by reminding users of past conversations. Alice cancels her coffee date with Bob when ElephantMind's service is down, fearing she won't remember their last chat. When Alice enthusiastically asks John about his trip to Paris, not knowing he was there for a difficult family matter, it leads to an awkward moment. Bob feels their friendship weakening as ElephantMind constantly pushes him to talk about trending political issues with Alice.}


\design{Civic Compass}{Civic Compass is an add-on for popular GPS navigation apps that highlights the ethical considerations of various routes, such as avoiding surveillance hotspots or environmentally protected habitats. Public policy debates have emerged over the app's potential abuse by criminals. When Alice is in a rush, she ignores its suggestions. She is also disturbed to see that most of the surveillance hotspots are in impoverished, marginalized neighborhoods.}

\design{BloodKey}{BloodKey is a form of two-factor authentication that can protect online accounts from stolen passwords. With BloodKey, after providing their password, users authenticate with a pin-prick blood test using an inexpensive biomedical Bluetooth gadget. BloodKey was quickly adopted by leading websites because it was very hard to fake, despite its invasive nature. However, Alice grew tired of the daily inconvenience of pricking her finger, so she gradually stopped logging into her social media accounts altogether, while Bob's medical condition prevented him from using it at all. A media frenzy ensued when the most popular free-to-play online game sold users' genomic data to pharmaceutical companies, boosting their revenue streams.}

\design{Profiled Plates} {Profiled Plates is a technology for high-end restaurants that offers an extraordinarily personalized dining experience by using extensive data mining and online profiling. With just a reservation name, the restaurant gathers detailed information about guests and their friends from social media, public records, and online interactions to create a bespoke meal tailored to individual preferences and backgrounds. Alice books a table for her birthday and is thrilled to find her favorite flowers on the table and a menu that excludes cilantro, a dislike she once tweeted about. However the system led to a low tip from Bob when it served a meal highly customized for a different Bob Bradley.}

\design{Square One}{Square One is a smart deletion feature for social media accounts which resets interests, deletes content, and unfollows people based on a simple prompt from the user about what they want removed. Jack uses Square One to unremember a volunteer project he led, only to find that potential employers now see him as less community-minded and pass him over for jobs. Jack is dismayed as he watches these projects falter under less-qualified hires.}

\subsection{Discussion}


\section{Methodology 3: LLM final}

\subsection{Methodology}

\subsection{Designs}

\design{Transparent Wallet}{Alice's digital wallet automatically categorizes her purchases and shares anonymized spending habits on a public leaderboard. She enjoys seeing her eco-friendly purchases earning her top marks in the green consumer category. However, Alice is troubled when she notices ads for luxury items appearing more frequently after her high spending months are tracked and shared by the wallet.}

\design{Echo Chamber Filter}{A social media app uses an advanced algorithm to filter content to match users' perceived political views and biases. At first, Alice appreciates the streamlined news feed with content she agrees with. Over time, she realizes her interactions are limited to a narrow worldview, reinforcing her existing beliefs and making it difficult to encounter differing perspectives or engage in meaningful discussions.}

\design{Emotion Tracker Wearable}{Alice’s smartwatch not only tracks her physical activity but also monitors her emotional states through heart rate variability and other biomarkers. The app suggests activities to boost her mood when she’s down. One day, Alice notices her insurance premium has increased. She discovers the data from her wearable has been shared with her insurance company, indicating high stress levels, which they consider a risk factor.}

\design{Algorithmic Hiring Bot}{Alice's company uses an AI-driven bot to screen job applications and conduct initial interviews. The bot, trained on historical data, tends to favor candidates with backgrounds similar to existing employees. This results in a homogeneous workplace lacking diversity. Alice starts questioning the fairness of the hiring process when a highly qualified candidate she referred is rejected without a clear reason.}

\design{Smart Mirror}{Alice's smart mirror provides personalized skincare recommendations and tracks changes in her appearance over time. It uses facial recognition to analyze her emotions and suggest products to improve her mood. One day, Alice notices the mirror suggesting makeup products she dislikes. She realizes the mirror has been selling her data to cosmetic companies, leading to targeted ads that influence her purchasing decisions.}

\section{Methodology 3: LLM Brainstorm}

\subsection{Methodology}

\subsection{Designs}

\paragraph{Virtual Persona} 
Virtual Persona is a social media platform that encourages users to create exaggerated online personas. Alice enjoys the freedom to express herself more boldly online but faces backlash when her boss finds her controversial posts, revealing the stark difference between acceptable behavior online and offline.

\paragraph{Minimal Access}
Minimal Access is a file-sharing app that strictly enforces the principle of least privilege, granting users only the minimal permissions required for their tasks. Alice appreciates the security but finds herself frustrated when she cannot access files she needs for urgent work, highlighting the challenges of balancing security and usability.

\paragraph{Data Trails}
Data Trails is a visualization tool that shows how Alice's data moves and interacts across various platforms after she shares it. Alice is shocked to see her data mingling with countless unknown entities, emphasizing how data takes on a life of its own once shared.

\paragraph{ShareSmart}
ShareSmart is a social media app that tracks and displays the amount of personal information Alice shares compared to her privacy concerns. Despite her vocal worries about privacy, the app shows that Alice continues to share significant amounts of personal data, highlighting the privacy paradox.

\paragraph{AnonGroup}
AnonGroup is a privacy-focused communication tool that groups users into anonymity sets to protect their identities. Alice enjoys the sense of security but grows uncomfortable when she realizes that within her anonymity set, some members are engaging in questionable activities, making her question the balance between privacy and accountability.

\paragraph{FreeFace}
FreeFace is a free photo-editing app that funds itself by selling user data to advertisers. Alice loves the app's features until she notices eerily personalized ads and learns her data is being sold, leading her to confront the reality that she is the product.

\paragraph{FuturePath}
FuturePath is a career guidance app that uses AI to predict Alice's career trajectory based on her current skills and behavior. Initially useful, Alice becomes uneasy when the app's deterministic predictions start influencing her choices, making her feel trapped by the technology's projections.

\paragraph{SafeLock}
SafeLock is a smart home security system that promises impenetrable protection. Alice feels secure until a minor system glitch leaves her home vulnerable, revealing her over-reliance on presumed security and the false sense of safety it provided.

\paragraph{BioPass}
BioPass is a biometric authentication app that uses facial recognition and fingerprints for access. Alice enjoys the convenience but becomes alarmed when a data breach exposes her biometric data, making her realize the unique and irreversible nature of such data compared to passwords.

\paragraph{CleanNet} 
CleanNet is a browser extension that gamifies cyber hygiene practices by awarding points for good habits. Alice starts using it enthusiastically but feels embarrassed when the app publicly ranks her low compared to her peers, driving home the importance of maintaining good cyber hygiene.

\subsection{Discussion}

\section{Concluding Remarks}




% = = = = = Ack = = = = = %

%\begin{anonsuppress}
%\begin{acks}
%We thank the reviewers who helped to improve our paper. 
%
%J. Clark acknowledges support for this research project from the \grantsponsor{nserc}{National Sciences and Engineering Research Council of Canada (NSERC)}{https://www.nserc-crsng.gc.ca} through the NSERC, Raymond Chabot Grant Thornton, and Catallaxy Industrial Research Chair in Blockchain Technologies \grantnum[https://www.nserc-crsng.gc.ca/Chairholders-TitulairesDeChaire/Chairholder-Titulaire_eng.asp?pid=1045]{nserc}{ IRCPJ/545498-2018} and a Discovery Grant \grantnum{nserc}{RGPIN/04019-2021}
%
%We also acknowledge partial support from 
%\grantsponsor{amf}{Autorité des Marchés Financiers (AMF)}{https://lautorite.qc.ca} and
%\grantsponsor{chaire}{The Chaire Fintech: AMF -- Finance Montréal}{https://chairefintech.uqam.ca} and
%\grantsponsor{opc}{Office of the Privacy Commissioner of Canada (OPC)}{https://www.priv.gc.ca}.
%
%\end{acks}
%\end{anonsuppress}

% = = = = = Bibliography = = = = = %
\bibliographystyle{ACM-Reference-Format}
\bibliography{bib/design.bib}

% = = = = = End Notes = = = = = %

\end{document}

