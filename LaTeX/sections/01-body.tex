% !TEX root = ../main.tex

\section{Introductory Remarks}

Blah, blah, blah

\section{Preliminaries}

% !TEX root = ../main.tex

\begin{table*}
\centering
\caption{Examples of Critical Design\label{tab:examples}}
\begin{tabular}{|p{3.5cm}|p{13cm}|} 
\hline

Work & Description \\ \hline

\textit{Compass Table}, \newline A. Dunne \newline F. Raby &
``This table reminds you that electronic objects extend beyond their visible limits. The 25 compasses set into its surface twitch and spin when objects like mobile phones or laptop computers are placed on it. The twitching needles can be interpreted as being either sinister or charming, depending on the viewer's state of mind. When we designed the compass table, we wondered if a neat-freak might try to make all the needles line up, ignoring the architectural space of the room in favour of the Earth's magnetic field.'' Quoted from \cite{DuRa01}. \\ \hline

\textit{Life Counter}, \newline I. Matsumoto &
``With Life Counter (2001), you choose how many years you would like to or expect to live for and start the counter. Once activated, it counts down the selected time span at four different rates: the number of years, days, hours or seconds to go are shown on different faces. Depending on which face you choose to display, you may feel very relaxed as the years stretch out ahead or begin to panic as you see your life speed away before your eyes. The counter is designed to be visually unassuming and could easily fit into the slightly retro-futuristic style of the moment.'' Quoted from \cite{DuRa01}. \\ \hline

\textit{Open Informant}, \newline J Ardern \newline Y. Ushigome \newline A. Jain &
``Open Informant (2013) takes the form of a networked object including a phone app and e-ink badge. The app searches your communications for National Security Agency trigger words and then sends text fragments containing these words to the badge worn by the user for public display. Using the body as an instrument for protest, the badge becomes a means of rendering our own voice visible in an otherwise faceless technological panopticon.'' Quoted from \cite{Mal17}. \\ \hline

\end{tabular}
\end{table*}

Critical design is commonly attributed to Dunne and Raby~\cite{DuRa01,Dun05} however it can be seen as a refinement of earlier movements in arts and design, beginning with the Italian radical design movement and extending through critical practice in HCI~\cite{Mal17}. Consider the three examples provided in Table~\ref{tab:examples}. Each describes the design of an object or technology. In contrast to industrial design, these technologies are not meant for mass production. Rather they offer a subtle criticism by illustrating something---electromagnetic radiation, the inevitability of death, the ubiquity and stealth of surveillance---that typically goes unquestioned, and in some cases this creates a dilemma for the user.

Critical designs are commonly described through vignettes. Occasionally they are built and displayed (see relation to art below). In the case of Dunne and Raby's Compass Table (and the seven other objects in their \textit{Placebo} project~\cite{DuRa01}), the table was built and given to individuals to use. This field study was followed by an exit survey on how the participants felt about the object. In this paper, we describe our designs through short vignettes that illustrate its design and use case. If they are a successful design, the reader should easily imagine its use in her life and question it critically, resulting in its `rhetorical use'~\cite{Mal17} rather than practical use. These designs are transgressive and intended to provoke the reader's emotions; invoking humour, satire, ridicule, poetry, playfulness, lewdness, appropriation, irreverence or deliberate overcompensation. We tag our own designs to highlight the intended mechanism.

\paragraph{Relation to Design.}

Much of design is oriented around problem-solving and the creation of solutions that are functional, useful, and pleasurable. In other words, design is affirmative. Critical design is about raising questions, challenging values, and stimulating debate. Critical design considers these goals to be the function (or `para-function'~\cite{Dun05}) of the concept, enabling traditional design approaches to maximizing this functionality. 

% Something about our approach it is relation to design

\paragraph{Relation to Art.} 

Critical design has significant overlap with conceptual art, utilizing common methods, and has been disseminated through exhibitions and the art literature. However critical design works by showcasing things that are plausible, believable, and easy for the reader to situate, requiring design practice. Viewing critical design as art also creates challenges. Consider the compass table in the context of an art gallery exhibit---it is meant to illustrate EMR in every day life, in one's home, in different positions and orientations within that home---all of this is lost when it sits statically in a gallery. When critical design is displayed in galleries, a vignette or film or narrative will often accompany the object to describe its context of use.

\paragraph{Relation to Science.} 

Critical design is ideological. Science can be as well. Even cryptography, positioned as one of the most scientifically rigorous pursuits in the field of security and privacy, is ideological, as argued eloquently by Rogaway in a recent keynote~\cite{Rog15}. For example, he notes that work on bland-sounding subjects like differential privacy and identity-based encryption are premised on a certain values and world views---respectively, one of massive data collection and one of government-escrowed master decryption keys. His point is that work in these areas \textit{affirm} certain ideologies; it is \textit{affirmative} design and more cryptographers should consider what is it that their designs affirm?

From a different angle, Herley and van Oorschot question which aspects of security, privacy, and cryptography are even a proper science~\cite{HvO17}. Critical design is certainly not an objective form of design where success can be measured. We purposely do not set out to `measure' the success of our designs as this deeply misses the point of critical design. While our designs are for scientists, engineers, and technologists, it is not to establish theorems or facts but to raise new questions and communicate risks and invasions in a new way.

\paragraph{Related Work}

Much has been written on critical design and Malpass provides an excellent survey~\cite{Mal17}. Since our focus is security and privacy, we only summarize the following concepts, situated within or close to critical design, since they deal with issues of privacy: 

\begin{compactlist}
\item \textbf{The Pillow.} Dunne and Gaver's design takes the form factor of a transparent pillow and plays radio signals it picks up from the technology of that time: \eg phones, pagers, and baby monitors~\cite{DuGa97}. It challenges users to consider how confidential these signals are.  
\item \textbf{Open Informant.} Described in Table~\ref{tab:examples}, this design exposes nation-state surveillance.
\item \textbf{United Mirco-Kingdoms.} Dunne and Raby reimagine the UK as four `supershires' where one is occupied by `digitarians:' extreme technocrats ruled by algorithms in a state of total surveillance.  
\item \textbf{Alternet.} While not critical design, Gold's Alternet has the similar intention of provoking thought through design~\cite{Gol14}. The internet is re-imagined with users retaining full control over their private data, touching on web tracking and our growing digital dossiers. 
\end{compactlist}

We are unaware of any attempt to build a systematic set of examples based on security or privacy. Many critical designs are `one-off.' The idea of developing a set of concepts related by a common theme, rather than individual concepts, is however established. For example, the Placebo Project mentioned already offered 8 concepts  (including Compass Table in Table~\ref{tab:examples}) that were each related to EMF radiation. 

Some security and privacy research offers seemingly unintentional critical design. In this case, we subsume the designs in our set of 101 while clearly noting when they originate from other researchers. We also include `found' examples of critical design from outside of academia. We believe that situating them alongside our original ideas allows the reader to consider how close our concepts are to reality. 

\section{A Selection of Concepts}

In this section, we present of selection of our 101 concepts. The complete set can be found in the full version of the paper. We curated these to illustrate different aspects of critical design and we accompany each vignette with some remarks on its design.

% = = = = = = = =

\paragraph{Dynamic Laughtrack}

Digital television content, such as sitcoms, encode laughtracks as a series of cues rather than an actual recording. Smart TVs listen and classify the viewer's level of laughter on a 10 point scale (with fine grain training over time). The TV adds laughter to the content in a senstive and considerate manner, where the laughtrack is only marginally higher on the scale than the viewer's own laughter. This gently nudges the viewer to greater enjoyment of the program without bombarding her. It also mixes in actual past recordings of the viewer's own laughter to capitalize on emotional mirroring.

\textit{Remarks:} derived from Alexis/Homepod design

\textit{Tags:} subtle;


